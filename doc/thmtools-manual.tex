%&pdflatex
\documentclass[paper=a4, pagesize, DIV=20,
  english, 
  headings=small,11pt,
  titlepage=false,
  numbers=noendperiod
]{scrreprt}
\addtokomafont{sectioning}{\rmfamily}

\usepackage[T1]{fontenc}
\usepackage[charter]{mathdesign}
\usepackage{luximono}
\usepackage[scaled]{helvet}

\usepackage[dvipsnames]{xcolor}

\usepackage{doc,shortvrb,xspace}
\makeatletter
\let\PrintDescribeMacro\@gobble
\let  \PrintDescribeEnv\@gobble
\let    \PrintMacroName\@gobble
\let      \PrintEnvName\@gobble
\makeatother

\usepackage{listings}

\usepackage{amsmath, amsthm}
\usepackage{tikz}

\usepackage{nameref}
\usepackage{hyperref}
\usepackage{cleveref}[2010/05/01]

\usepackage{thmtools, thm-restate}
\usepackage[unq]{unique}

\providecommand\pkg[1]{\textsf{#1}}
\providecommand\Thmtools{\pkg{Thmtools}\xspace}
\providecommand\thmtools{\pkg{thmtools}\xspace}
\providecommand\oarg[1]{\texttt{[}\textit{#1}\texttt{]}}
\providecommand\marg[1]{\texttt{\{}\textit{#1}\texttt{\}}}


\providecommand\clap{\makebox[0pt][c]}

\lstloadlanguages{[LaTeX]TeX}
\lstset{language=[LaTeX]TeX,basicstyle=\small\ttfamily,keywordstyle=\mdseries,aboveskip=0pt}

\lstnewenvironment{preamble}[1][]{%
  \lstset{backgroundcolor=\color{Purple!15},#1}%
}{%
}
\lstnewenvironment{body}[1][]{%
  \lstset{backgroundcolor=\color{Yellow!30},#1}%
}{%
}

\newenvironment{source}{%
  \par\noindent\strut\minipage[t]{0.61\linewidth}
}{%
  \endminipage
}
\newenvironment{result}{%
  \hfill\minipage[t]{0.37\linewidth}
}{%
  \endminipage\strut\par
}
\MakeShortVerb{\|}


\declaretheorem{theorem}
\declaretheorem[numberwithin=section]{theoremS}
\declaretheorem[name=\"Ubung]{exercise}
\declaretheorem[sibling=theorem]{lemma}
\declaretheorem[numbered=no,
  name=Euclid's Prime Theorem]{euclid}
\declaretheorem[numbered=unless unique]{singleton}
\declaretheorem[numbered=unless unique]{couple}
\declaretheorem[style=remark]{remark}
\declaretheorem{Theorem}

\declaretheorem[shaded={bgcolor=Lavender,
      textwidth=12em}]{BoxI}

\declaretheorem[shaded={rulecolor=Lavender,
  rulewidth=2pt, bgcolor={rgb}{1,1,1}}]{BoxII}

    \declaretheorem[thmbox=L]{boxtheorem L}
    \declaretheorem[thmbox=M]{boxtheorem M}
    \declaretheorem[thmbox=S]{boxtheorem S}

\declaretheoremstyle[
      spaceabove=6pt, spacebelow=6pt,
      headfont=\normalfont\bfseries,
      notefont=\mdseries, notebraces={(}{)},
      postheadspace=1em,
      qed=\qedsymbol
    ]{mystyle}
\declaretheorem[style=mystyle
  ]{styledtheorem}

\declaretheorem[name=Theorem, refname={theorem,theorems},
    Refname={Theorem,Theorems}]{callmeal}


\def\x #1-#2-#3 #4 v#5\@{\def\VERSION{#1/#2/#3 v#5}}\x 2019-07-31 00:00:00 +0200 v67\@


\title{\Thmtools Users' Guide}
\author{Ulrich M. Schwarz -- ulmi@absatzen.de\thanks{
  who would like to thank the users for testing, encouragement, feature requests, and
  bug reports. In particular, Denis Bitouz\'e prompted further improvement
  when \thmtools got stuck in a ``good enough for me'' slump.
}}
\date{\VERSION}

\begin{document}
  \maketitle
  \section*{\abstractname}
  
  The \thmtools bundle is a collection of packages that is designed to
  provide an easier interface to theorems, and to facilitate some more
  advanced tasks.
  
  If you are a first-time user and you don't think your requirements are out
  of the ordinary, browse the examples in \autoref{cha:impatient}. If you're
  here because the other packages you've tried so far just can't do what you
  want, take inspiration from \autoref{cha:extravagant}. If you're a repeat
  customer, you're most likely to be interested in the refence section in
  \autoref{cha:reference}.
  
  \begin{multicols}{2}[\section*{\contentsname}]
  \makeatletter\let\chapter\@gobbletwo
  \tableofcontents
  \end{multicols}
  \clearpage
  
  \chapter{\Thmtools for the impatient}\label{cha:impatient}
  
  \section*{How to use this document}
  
  This guide consists mostly of examples and their output, sometimes with a
  few additional remarks. Since theorems are defined in the preamble and
  used in the document, the snippets are two-fold:
  \begin{source}
  \begin{preamble}[gobble=4]
    % Preamble code looks like this.
    \usepackage{amsthm}
    \usepackage{thmtools}
    \declaretheorem{theorem}
  \end{preamble}
  \begin{body}[gobble=4]
    % Document code looks like this.
    \begin{theorem}[Euclid]
     \label{thm:euclid}%
     For every prime $p$, there is a prime $p'>p$.
     In particular, the list of primes,
     \begin{equation}\label{eq:1}
       2,3,5,7,\dots
     \end{equation}
     is infinite.
    \end{theorem}
  \end{body}
  \end{source}
  \begin{result}
    The result looks like this:
%    \begin{theorem}[Euclid]
%      For every prime $p$, there is a prime $p'>p$.
%      In particular, there are infinitely many primes.
%    \end{theorem}
  \begin{restatable}[Euclid]{theorem}{firsteuclid}
     \label{thm:euclid}%
     For every prime $p$, there is a prime $p'>p$.
     In particular, the list of primes,
     \begin{equation}\label{eq:1}
       2,3,5,7,\dots
     \end{equation}
     is infinite.
  \end{restatable}
  \end{result}
  
  Note that in all cases, you will need a \emph{backend} to provide the
  command \lstinline|\newtheorem| with the usual behaviour. The \LaTeX\
  kernel has a built-in backend which cannot do very much; the most common
  backends these days are the amsthm and ntheorem packages. Throughout this
  document, we'll use amsthm.
  
  \section{Elementary definitions}
  
  As you have seen above, the new command to define theorems is
  \lstinline|\declaretheorem|, which in its most basic form just takes the
  name of the environment. All other options can be set through a key-val
  interface:
  \begin{source}
  \begin{preamble}[gobble=4]
    \usepackage{amsthm}
    \usepackage{thmtools}
    \declaretheorem[numberwithin=section]{theoremS}
  \end{preamble}
  \begin{body}[gobble=4]
    \begin{theoremS}[Euclid]
      For every prime $p$, there is a prime $p'>p$.
      In particular, there are infinitely many primes.
    \end{theoremS}
  \end{body}
  \end{source}
  \begin{result}
    \begin{restatable}[Euclid]{theoremS}{euclidii}
      For every prime $p$, there is a prime $p'>p$.
      In particular, there are infinitely many primes.
    \end{restatable}
  \end{result}
  
  Instead of ``numberwithin='', you can also use ``parent='' and
  ``within=''. They're all the same, use the one you find easiest to
  remember.
  
  Note the example above looks somewhat bad: sometimes, the name of the environment, 
  with the first
  letter uppercased, is not a good choice for the theorem's title.
  
  \begin{source}
  \begin{preamble}[gobble=4]
    \usepackage{amsthm}
    \usepackage{thmtools}
    \declaretheorem[name=\"Ubung]{exercise}
  \end{preamble}
  \begin{body}[gobble=4]
    \begin{exercise}
      Prove Euclid's Theorem.
    \end{exercise}
  \end{body}
  \end{source}
  \begin{result}
    \begin{exercise}
      Prove Euclid's Theorem.
    \end{exercise}
  \end{result}
  
  To save you from having to look up the name of the key every time, you can
  also use ``title='' and ``heading='' instead of ``name=''; they do exactly
  the same and hopefully one of these will be easy to remember for you.

  Of course, you do not have to follow the abominal practice of numbering
  theorems, lemmas, etc., separately:  
  \begin{source}
  \begin{preamble}[gobble=4]
    \usepackage{amsthm}
    \usepackage{thmtools}
    \declaretheorem[sibling=theorem]{lemma}
  \end{preamble}
  \begin{body}[gobble=4]
    \begin{lemma}
      For every prime $p$, there is a prime $p'>p$.
      In particular, there are infinitely many primes.      
    \end{lemma}
  \end{body}
  \end{source}
  \begin{result}
    \begin{lemma}
      For every prime $p$, there is a prime $p'>p$.
      In particular, there are infinitely many primes.      
    \end{lemma}
  \end{result}

  Again, instead of ``sibling='', you can also use ``numberlike='' and
  ``sharecounter=''.

  Some theorems have a fixed name and are not supposed to get a number.
  To this end, amsthm provides \lstinline|\newtheorem*|, which is
  accessible through thmtools:
  \begin{source}
  \begin{preamble}[gobble=4]
    \usepackage{amsthm}
    \usepackage{thmtools}
    \declaretheorem[numbered=no,
      name=Euclid's Prime Theorem]{euclid}
  \end{preamble}
  \begin{body}[gobble=4]
    \begin{euclid}
      For every prime $p$, there is a prime $p'>p$.
      In particular, there are infinitely many primes.      
    \end{euclid}
  \end{body}
  \end{source}
  \begin{result}
    \begin{euclid}
      For every prime $p$, there is a prime $p'>p$.
      In particular, there are infinitely many primes.      
    \end{euclid}
  \end{result}

  As a somewhat odd frill, you can turn off the number if there's only one
  instance of the kind in the document. This might happen when you split and
  join your papers into short conference versions and longer journal papers
  and tech reports. Note that this doesn't combine well with the sibling
  key: how do you count like somebody who suddenly doesn't count anymore?
  Also, it takes an extra \LaTeX\ run to settle.
  \begin{source}
  \begin{preamble}[gobble=4]
    \usepackage{amsthm}
    \usepackage{thmtools}
    \usepackage[unq]{unique}
    \declaretheorem[numbered=unless unique]{singleton}
    \declaretheorem[numbered=unless unique]{couple}
  \end{preamble}
  \begin{body}[gobble=4]
    \begin{couple}
      Marc \& Anne
    \end{couple}
    \begin{singleton}
      Me.
    \end{singleton}
    \begin{couple}
      Buck \& Britta
    \end{couple}
  \end{body}
  \end{source}
  \begin{result}
    \begin{couple}
      Marc \& Anne
    \end{couple}
    \begin{singleton}
      Me.
    \end{singleton}
    \begin{couple}
      Buck \& Britta
    \end{couple}
  \end{result}

  \section{Frilly references}

  In case you didn't know, you should: \pkg{hyperref}, \pkg{nameref} and
  \pkg{cleveref} offer ways of ``automagically'' knowing that
  \lstinline|\label{foo}| was inside a theorem, so that a reference adds the
  string ``Theorem''. This is all done for you, but there's one catch: you
  have to tell \thmtools\ what the name to add is. (In singular and plural
  form for \pkg{cleveref}.
  \begin{source}
    \begin{preamble}[gobble=6]
      \usepackage{amsthm, thmtools}
      \usepackage{
        nameref,%\nameref
        hyperref%\autoref; 
        % n.b. \Autoref is defined by thmtools
        cleveref,% \cref
        % nb cleveref after! hyperref
      }
      \declaretheorem[name=Theorem, 
        refname={theorem,theorems},
        Refname={Theorem,Theorems}]{callmeal}
    \end{preamble}
    \begin{body}[gobble=6]
      \begin{callmeal}[Simon]\label{simon}
        One
      \end{callmeal}
      \begin{callmeal}\label{garfunkel}
        and another, and together, 
        \autoref{simon}, ``\nameref{simon}'',
        and \cref{garfunkel} are referred 
        to as \cref{simon,garfunkel}.
        \Cref{simon,garfunkel}, if you are at 
        the beginning of a sentence.
      \end{callmeal}
    \end{body}
  \end{source}
  \begin{result}
      \begin{callmeal}[Simon]\label{simon}
        One
      \end{callmeal}
      \begin{callmeal}\label{garfunkel}
        and another, and together, \autoref{simon}, ``\nameref{simon}'',
        and \cref{garfunkel} are referred to as \cref{simon,garfunkel}.
        \Cref{simon,garfunkel}, if you are at the beginning of a sentence.
      \end{callmeal}
  \end{result}

  \section{Styling theorems}
  
  The major backends provide a command \lstinline|\theoremstyle| to switch
  between looks of theorems. This is handled as follows:
  \begin{source}
  \begin{preamble}[gobble=4]
    \usepackage{amsthm}
    \usepackage{thmtools}
    \declaretheorem[style=remark]{remark}
    \declaretheorem{Theorem}
  \end{preamble}
  \begin{body}[gobble=4]
    \begin{Theorem}
      This is a theorem.
    \end{Theorem}
    \begin{remark}
      Note how it still retains the default style, `plain'.
    \end{remark}
  \end{body}
  \end{source}
  \begin{result}
    \begin{Theorem}
      This is a theorem.
    \end{Theorem}
    \begin{remark}
      Note how it still retains the default style, `plain'.
    \end{remark}
  \end{result}
  
  Thmtools also supports the shadethm and thmbox packages:
  \begin{source}
  \begin{preamble}[gobble=4]
    \usepackage{amsthm}
    \usepackage{thmtools}
    \usepackage[dvipsnames]{xcolor}
    \declaretheorem[shaded={bgcolor=Lavender,
      textwidth=12em}]{BoxI}
    \declaretheorem[shaded={rulecolor=Lavender,
      rulewidth=2pt, bgcolor={rgb}{1,1,1}}]{BoxII}
  \end{preamble}
  \begin{body}[gobble=4]
    \begin{BoxI}[Euclid]
      For every prime $p$, there is a prime $p'>p$.
      In particular, there are infinitely many primes.
    \end{BoxI}
    \begin{BoxII}[Euclid]
      For every prime $p$, there is a prime $p'>p$.
      In particular, there are infinitely many primes.
    \end{BoxII}
  \end{body}
  \end{source}
  \begin{result}
    \begin{BoxI}
      For every prime $p$, there is a prime $p'>p$.
      In particular, there are infinitely many primes.
    \end{BoxI}
    \begin{BoxII}
      For every prime $p$, there is a prime $p'>p$.
      In particular, there are infinitely many primes.
    \end{BoxII}
  \end{result}

  As you can see, the color parameters can take two forms: it's either the
  name of a color that is already defined, without curly braces, or it can
  start with a curly brace, in which case it is assumed that
  \lstinline|\definecolor{colorname}|$\langle$\textsl{what you said}$\rangle$ will be
  valid \LaTeX\ code. In our case, we use the rbg model to manually specify
  white. (Shadethm's default value is some sort of gray.)
  
  For the thmbox package, use the thmbox key:
  \begin{source}
  \begin{preamble}[gobble=4]
    \usepackage{amsthm}
    \usepackage{thmtools}
    \declaretheorem[thmbox=L]{boxtheorem L}
    \declaretheorem[thmbox=M]{boxtheorem M}
    \declaretheorem[thmbox=S]{boxtheorem S}
  \end{preamble}
  \begin{body}[gobble=4]
    \begin{boxtheorem L}[Euclid]
      For every prime $p$, there is a prime $p'>p$.
      In particular, there are infinitely many primes.
    \end{boxtheorem L}
    \begin{boxtheorem M}[Euclid]
      For every prime $p$, there is a prime $p'>p$.
      In particular, there are infinitely many primes.
    \end{boxtheorem M}
    \begin{boxtheorem S}[Euclid]
      For every prime $p$, there is a prime $p'>p$.
      In particular, there are infinitely many primes.
    \end{boxtheorem S}
  \end{body}
  \end{source}
  \begin{result}
    \begin{boxtheorem L}[Euclid]
      For every prime $p$, there is a prime $p'>p$.
      In particular, there are infinitely many primes.
    \end{boxtheorem L}
    \begin{boxtheorem M}[Euclid]
      For every prime $p$, there is a prime $p'>p$.
      In particular, there are infinitely many primes.
    \end{boxtheorem M}
    \begin{boxtheorem S}[Euclid]
      For every prime $p$, there is a prime $p'>p$.
      In particular, there are infinitely many primes.
    \end{boxtheorem S}
  \end{result}

  Note that for both thmbox and shaded keys, it's quite possible they will not
  cooperate with a style key you give at the same time.

  \subsection{Declaring new theoremstyles}
  
  \Thmtools\ also offers a new command to define new theoremstyles. It is
  partly a frontend to the \lstinline|\newtheoremstyle| command of amsthm or
  ntheorem, but it offers (more or less successfully) the settings of both to 
  either. So we are talking about the same things, consider the sketch in
  \autoref{fig:params}. To get a result like that, you would use something
  like
  \begin{source}
  \begin{preamble}[gobble=4]
    \declaretheoremstyle[
      spaceabove=6pt, spacebelow=6pt,
      headfont=\normalfont\bfseries,
      notefont=\mdseries, notebraces={(}{)},
      postheadspace=1em,
      qed=\qedsymbol
    ]{mystyle}
    \declaretheorem[style=mystyle]{styledtheorem}
  \end{preamble}
  \begin{body}[gobble=4]
    \begin{styledtheorem}[Euclid]
      For every prime $p$\dots
    \end{styledtheorem}
  \end{body}
  \end{source}
  \begin{result}
    \begin{styledtheorem}[Euclid]
      For every prime $p$\dots
    \end{styledtheorem}
  \end{result}
  Again, the defaults are reasonable and you don't
  have to give values for everything.
  
  There is one important thing you cannot see in this example: there are
  more keys you can pass to \lstinline|\declaretheoremstyle|: if \thmtools\ cannot
  figure out at all what to do with it, it will pass it on to the
  \lstinline|\declaretheorem| commands that use that style. For example, you may use
  the boxed and shaded keys here.
  
  To change the order in which title, number and note appear, there is a key
  headstyle. Currently, the values ``margin'' and ``swapnumber'' are
  supported. The daring may also try to give a macro here that uses the
  commands \lstinline|\NUMBER|, \lstinline|\NAME| and \lstinline|\NOTE|. 
  You cannot circumvent the fact
  that headpunct comes at the end, though, nor the fonts and braces you
  select with the other keys.
  
  \begin{figure}\centering
    % please don't make me touch this picture ever again.
    \fbox{
    \begin{minipage}{0.618\textwidth}\Large
    \tikzset{font=\normalfont\small\sffamily\itshape,y=12pt,>=latex}
    \noindent which resulted in the following insight:
    \par
    \tikz{\draw[|<->|] (0,-1)--(0,1); 
      \draw[anchor=west] (0,0) node {spaceabove};}
    \par
    \tikz{\draw[|<->|] (-1,0)--(1,0); 
      \draw[anchor=south] (0,0) node {headindent};}
    \textbf{Theo%
    \smash{\clap{\tikz{\draw (0,0)--(0,1.2) node [anchor=south]{headfont};}}}%
      rem 1.2 
      (\smash{\rlap{\tikz{\draw (0,0)--(0,2.5) node [anchor=base west]{notebraces};}}}%
       Euc%
\smash{\clap{\tikz{\draw (0,0)--(0,1.2) node [anchor=south]{notefont};}}}%
       lid)%
      .\smash{\rlap{\tikz{\draw (0,0)--(0,1.2) node [anchor=south west]{headpunct};}}}}%
      \tikz{\draw[|<->|] (-1,0)--(1,0); 
      \draw[anchor=south] (0,0) node {postheadspace};}
    For every prime $p$, there is a prime $p'>p$.
    In particular, the list of primes,
    $2,3,5,7,\dots$,
    is infinite.
   \hfill{\tikz{\draw[anchor=north east] (0,0) node{qed}; }}$\Box$
    \par
    \tikz{\draw[|<->|] (0,-1)--(0,1); 
      \draw[anchor=west] (0,0) node {spacebelow};}
    \par
   
   As a consequence, lorem ipsum dolor sit amet frob-%nicate foo
%   paret.
  \end{minipage}
  }
    \caption{Settable parameters of a theorem style.}
    \label{fig:params}
  \end{figure}

  \section{Repeating theorems}
  
  Sometimes, you want to repeat a theorem you have given in full earlier,
  for example you either want to state your strong result in the
  introduction and then again in the full text, or you want to re-state a
  lemma in the appendix where you prove it. For example, I lied about
  \autoref{thm:euclid} on p.\,\pageref{thm:euclid}: the true code used was
  \begin{source}
    \begin{preamble}[gobble=6]
      \usepackage{thmtools, thm-restate}
      \declaretheorem{theorem}
    \end{preamble}
    \begin{body}[gobble=6]
      \begin{restatable}[Euclid]{theorem}{firsteuclid}
        \label{thm:euclid}%
        For every prime $p$, there is a prime $p'>p$.
        In particular, the list of primes,
        \begin{equation}\label{eq:1}
          2,3,45,7,\dots
        \end{equation}
        is infinite.
      \end{restatable}
    \end{body}
    and to the right, I just use
    \begin{body}[gobble=6]
      \firsteuclid*
      \vdots
      \firsteuclid*
    \end{body}
  \end{source}
  \begin{result}
    \firsteuclid*
    \vdots
    \firsteuclid*
  \end{result}
  
  Note that in spite of being a theorem-environment, it gets number one all
  over again. Also, we get equation number~\eqref{eq:1} again. The star in 
  \lstinline|\firsteuclid*| tells thmtools that it should redirect the label
  mechanism, so that this reference: \autoref{thm:euclid} points to
  p.\,\pageref{thm:euclid}, where the unstarred environment is used. (You can
  also use a starred environment and an unstarred command, in which case the
  behaviour is reversed.) Also, if you use hyperref, the links will lead you
  to the unstarred occurence.
  
  Just to demonstrate that we also handle more involved cases, I repeat
  another theorem here, but this one was numbered within its section: note
  we retain the section number which does not fit the current section:
  \begin{source}
    \begin{body}
      \euclidii*
    \end{body}
  \end{source}
  \begin{result}
    \euclidii*
  \end{result}

  
  \section{Lists of theorems}
  
  To get a list of theorems with default formatting, just use
  \lstinline|\listoftheorems|:
  \begin{source}
    \begin{body}[gobble=6]
      \listoftheorems
    \end{body}
  \end{source}
  \begin{result}
    \let\chapter\section
    \let\clearpage\relax
    \listoftheorems
  \end{result}
  
  Not everything might be of the same importance, so you can filter out
  things by environment name:
  \begin{source}
    \begin{body}[gobble=6]
      \listoftheorems[ignoreall, 
        show={theorem,Theorem,euclid}]
    \end{body}
  \end{source}
  \begin{result}
    \let\chapter\section
    \let\clearpage\relax
    \listoftheorems[ignoreall, show={theorem,Theorem,euclid}]
  \end{result}
  
  And you can also restrict to those environments that have an optional
  argument given. Note that two theorems disappear compared to the previous
  example. You could also say just ``onlynamed'', in which case it will
  apply to \emph{all} theorem environments you have defined.
  \begin{source}
    \begin{body}[gobble=6]
      \listoftheorems[ignoreall, 
        onlynamed={theorem,Theorem,euclid}]
    \end{body}
  \end{source}
  \begin{result}
    \let\chapter\section
    \let\clearpage\relax
    \listoftheorems[ignoreall, onlynamed={theorem,Theorem,euclid}]
  \end{result}

  As might be expected, the heading given is defined in \lstinline|\listoftheoremname|.

  \chapter{\Thmtools for the extravagant}\label{cha:extravagant}

  This chapter will go into detail on the slightly more technical offerings
  of this bundle. In particular, it will demonstrate how to use the general
  hooks provided to extend theorems in the way you want them to behave.
  Again, this is done mostly by some examples.
  
  \section{Understanding \thmtools' extension mechanism}

  TBD.

  \section{Case in point: the thmbox key}

  \DocInput{thmdef-thmbox.dtx}


  \section{Case in point: the shaded key}
  
  \DocInput{thmdef-shaded.dtx}
  

  \chapter{\Thmtools for the completionist}\label{cha:reference}

  This will eventually contain a reference to all known keys, commands, etc.


  \appendix
  

  \chapter{\Thmtools for the morbidly curious}\label{cha:sourcecode}

  This chapter consists of the implementation of Thmtools, in case you
  wonder how this or that feature was implemented. Read on if you want a
  look under the bonnet, but you enter at your own risk, and bring an oily
  rag with you.

  \section{Core functionality}
  
  \subsection{The main package}
  \DocInput{thmtools.dtx}

  \subsection{Adding hooks to the relevant commands}
  \DocInput{thm-patch.dtx}

  \subsection{The key-value interfaces}
  \DocInput{thm-kv.dtx}

  \subsection{Lists of theorems}
  \DocInput{thm-listof.dtx}

  \subsection{Re-using environments}
  \DocInput{thm-restate.dtx}

  \subsection{Fixing autoref and friends}  
  \DocInput{thm-autoref.dtx}

  \section{Glue code for different backends}
  
  \subsection{amsthm}
  \DocInput{thm-amsthm.dtx}

  \subsection{beamer}
  \DocInput{thm-beamer.dtx}

  \subsection{ntheorem}
  \DocInput{thm-ntheorem.dtx}
  
  \section{Generic tools}
  
  \subsection{A generalized argument parser}
  \DocInput{parseargs.dtx}
  
  \subsection{Different counters sharing the same register}
  \DocInput{aliasctr.dtx}
  
  \subsection{Tracking occurences: none, one or many}
  \DocInput{unique.dtx}

  

      
\end{document}
