%&pdflatex
\documentclass[paper=a4, pagesize, DIV=20,
  english, 
  headings=small,11pt,
  titlepage=false,
  numbers=noendperiod
]{scrreprt}
\addtokomafont{sectioning}{\rmfamily}

\usepackage[T1]{fontenc}
\usepackage[charter]{mathdesign}
\usepackage{luximono}
\usepackage[scaled]{helvet}

\usepackage[dvipsnames]{xcolor}

\usepackage{doc,shortvrb,xspace}
\makeatletter
% to be fixed at some point...
\let\PrintDescribeMacro\@gobble
\let  \PrintDescribeEnv\@gobble
\let    \PrintMacroName\@gobble
\let      \PrintEnvName\@gobble
\makeatother
\CodelineNumbered

\usepackage{listings}

\usepackage{amsmath, amsthm}
\usepackage{tikz}

\usepackage{nameref}
\usepackage{hyperref}
\usepackage{cleveref}[2010/05/01]

\usepackage{thmtools, thm-restate}
\usepackage[unq]{unique}

\providecommand\pkg[1]{\textsf{#1}}
\providecommand\Thmtools{\pkg{Thmtools}\xspace}
\providecommand\thmtools{\pkg{thmtools}\xspace}
\providecommand\oarg[1]{\texttt{[}\textit{#1}\texttt{]}}
\providecommand\marg[1]{\texttt{\{}\textit{#1}\texttt{\}}}


\providecommand\clap{\makebox[0pt][c]}

\newcommand\key[2][\keydefaultcontext]{%
  % todo: create an index from this...
  \paragraph*{\texttt{#2}}
}

\lstloadlanguages{[LaTeX]TeX}
\lstset{language=[LaTeX]TeX,basicstyle=\small\ttfamily,keywordstyle=\mdseries,aboveskip=0pt}

\lstnewenvironment{preamble}[1][]{%
  \lstset{backgroundcolor=\color{Purple!15},#1}%
}{%
}
\lstnewenvironment{body}[1][]{%
  \lstset{backgroundcolor=\color{Yellow!30},#1}%
}{%
}

\newenvironment{source}{%
  \par\noindent\strut\minipage[t]{0.61\linewidth}
}{%
  \endminipage
}
\newenvironment{result}{%
  \hfill\minipage[t]{0.37\linewidth}
}{%
  \endminipage\strut\par
}
%\MakeShortVerb{\|}
\lstMakeShortInline{|}


\declaretheorem{theorem}
\declaretheorem[numberwithin=section]{theoremS}
\declaretheorem[name=\"Ubung]{exercise}
\declaretheorem[sibling=theorem]{lemma}
\declaretheorem[numbered=no,
  name=Euclid's Prime Theorem]{euclid}
\declaretheorem[numbered=unless unique]{singleton}
\declaretheorem[numbered=unless unique]{couple}
\declaretheorem[style=remark]{remark}
\declaretheorem{Theorem}

\declaretheorem[shaded={bgcolor=Lavender,
      textwidth=12em}]{BoxI}

\declaretheorem[shaded={rulecolor=Lavender,
  rulewidth=2pt, bgcolor={rgb}{1,1,1}}]{BoxII}

    \declaretheorem[thmbox=L]{boxtheorem L}
    \declaretheorem[thmbox=M]{boxtheorem M}
    \declaretheorem[thmbox=S]{boxtheorem S}

\declaretheoremstyle[
      spaceabove=6pt, spacebelow=6pt,
      headfont=\normalfont\bfseries,
      notefont=\mdseries, notebraces={(}{)},
      bodyfont=\normalfont,
      postheadspace=1em,
      qed=\qedsymbol
    ]{mystyle}
\declaretheorem[style=mystyle
  ]{styledtheorem}

\declaretheorem[name=Theorem, refname={theorem,theorems},
    Refname={Theorem,Theorems}]{callmeal}


\def\x #1-#2-#3 #4 v#5\@{\def\VERSION{#1/#2/#3 v#5}}\x 2019-07-31 00:00:00 +0200 v67\@


\title{\Thmtools Users' Guide}
\author{Dr. Ulrich M. Schwarz -- ulmi@absatzen.de\thanks{
  who would like to thank the users for testing, encouragement, feature requests, and
  bug reports. In particular, Denis Bitouz\'e prompted further improvement
  when \thmtools got stuck in a ``good enough for me'' slump.
}}
\date{\VERSION}

\begin{document}
  \maketitle
  \section*{\abstractname}
  
  The \thmtools bundle is a collection of packages that is designed to
  provide an easier interface to theorems, and to facilitate some more
  advanced tasks.
  
  If you are a first-time user and you don't think your requirements are out
  of the ordinary, browse the examples in \autoref{cha:impatient}. If you're
  here because the other packages you've tried so far just can't do what you
  want, take inspiration from \autoref{cha:extravagant}. If you're a repeat
  customer, you're most likely to be interested in the refence section in
  \autoref{cha:reference}.
  
  \begin{multicols}{2}[\section*{\contentsname}]
  \makeatletter\let\chapter\@gobbletwo
  \tableofcontents
  \end{multicols}
  \clearpage
  
  \chapter{\Thmtools for the impatient}\label{cha:impatient}
  
  \section*{How to use this document}
  
  This guide consists mostly of examples and their output, sometimes with a
  few additional remarks. Since theorems are defined in the preamble and
  used in the document, the snippets are two-fold:
  \begin{source}
  \begin{preamble}[gobble=4]
    % Preamble code looks like this.
    \usepackage{amsthm}
    \usepackage{thmtools}
    \declaretheorem{theorem}
  \end{preamble}
  \begin{body}[gobble=4]
    % Document code looks like this.
    \begin{theorem}[Euclid]
     \label{thm:euclid}%
     For every prime $p$, there is a prime $p'>p$.
     In particular, the list of primes,
     \begin{equation}\label{eq:1}
       2,3,5,7,\dots
     \end{equation}
     is infinite.
    \end{theorem}
  \end{body}
  \end{source}
  \begin{result}
    The result looks like this:
%    \begin{theorem}[Euclid]
%      For every prime $p$, there is a prime $p'>p$.
%      In particular, there are infinitely many primes.
%    \end{theorem}
  \begin{restatable}[Euclid]{theorem}{firsteuclid}
     \label{thm:euclid}%
     For every prime $p$, there is a prime $p'>p$.
     In particular, the list of primes,
     \begin{equation}\label{eq:1}
       2,3,5,7,\dots
     \end{equation}
     is infinite.
  \end{restatable}
  \end{result}
  
  Note that in all cases, you will need a \emph{backend} to provide the
  command |\newtheorem| with the usual behaviour. The \LaTeX\
  kernel has a built-in backend which cannot do very much; the most common
  backends these days are the \pkg{amsthm} and \pkg{ntheorem} packages. 
  Throughout this document, we'll use \pkg{amsthm}, and some of the features
  won't work with \pkg{ntheorem}.
  
  \section{Elementary definitions}
  
  As you have seen above, the new command to define theorems is
  |\declaretheorem|, which in its most basic form just takes the
  name of the environment. All other options can be set through a key-val
  interface:
  \begin{source}
  \begin{preamble}[gobble=4]
    \usepackage{amsthm}
    \usepackage{thmtools}
    \declaretheorem[numberwithin=section]{theoremS}
  \end{preamble}
  \begin{body}[gobble=4]
    \begin{theoremS}[Euclid]
      For every prime $p$, there is a prime $p'>p$.
      In particular, there are infinitely many primes.
    \end{theoremS}
  \end{body}
  \end{source}
  \begin{result}
    \begin{restatable}[Euclid]{theoremS}{euclidii}
      For every prime $p$, there is a prime $p'>p$.
      In particular, there are infinitely many primes.
    \end{restatable}
  \end{result}
  
  Instead of ``numberwithin='', you can also use ``parent='' and
  ``within=''. They're all the same, use the one you find easiest to
  remember.
  
  Note the example above looks somewhat bad: sometimes, the name of the environment, 
  with the first
  letter uppercased, is not a good choice for the theorem's title.
  
  \begin{source}
  \begin{preamble}[gobble=4]
    \usepackage{amsthm}
    \usepackage{thmtools}
    \declaretheorem[name=\"Ubung]{exercise}
  \end{preamble}
  \begin{body}[gobble=4]
    \begin{exercise}
      Prove Euclid's Theorem.
    \end{exercise}
  \end{body}
  \end{source}
  \begin{result}
    \begin{exercise}
      Prove Euclid's Theorem.
    \end{exercise}
  \end{result}
  
  To save you from having to look up the name of the key every time, you can
  also use ``title='' and ``heading='' instead of ``name=''; they do exactly
  the same and hopefully one of these will be easy to remember for you.

  Of course, you do not have to follow the abominal practice of numbering
  theorems, lemmas, etc., separately:  
  \begin{source}
  \begin{preamble}[gobble=4]
    \usepackage{amsthm}
    \usepackage{thmtools}
    \declaretheorem[sibling=theorem]{lemma}
  \end{preamble}
  \begin{body}[gobble=4]
    \begin{lemma}
      For every prime $p$, there is a prime $p'>p$.
      In particular, there are infinitely many primes.      
    \end{lemma}
  \end{body}
  \end{source}
  \begin{result}
    \begin{lemma}
      For every prime $p$, there is a prime $p'>p$.
      In particular, there are infinitely many primes.      
    \end{lemma}
  \end{result}

  Again, instead of ``sibling='', you can also use ``numberlike='' and
  ``sharecounter=''.

  Some theorems have a fixed name and are not supposed to get a number.
  To this end, \pkg{amsthm} provides |\newtheorem*|, which is
  accessible through \thmtools:
  \begin{source}
  \begin{preamble}[gobble=4]
    \usepackage{amsthm}
    \usepackage{thmtools}
    \declaretheorem[numbered=no,
      name=Euclid's Prime Theorem]{euclid}
  \end{preamble}
  \begin{body}[gobble=4]
    \begin{euclid}
      For every prime $p$, there is a prime $p'>p$.
      In particular, there are infinitely many primes.      
    \end{euclid}
  \end{body}
  \end{source}
  \begin{result}
    \begin{euclid}
      For every prime $p$, there is a prime $p'>p$.
      In particular, there are infinitely many primes.      
    \end{euclid}
  \end{result}

  As a somewhat odd frill, you can turn off the number if there's only one
  instance of the kind in the document. This might happen when you split and
  join your papers into short conference versions and longer journal papers
  and tech reports. Note that this doesn't combine well with the sibling
  key: how do you count like somebody who suddenly doesn't count anymore?
  Also, it takes an extra \LaTeX\ run to settle.
  \begin{source}
  \begin{preamble}[gobble=4]
    \usepackage{amsthm}
    \usepackage{thmtools}
    \usepackage[unq]{unique}
    \declaretheorem[numbered=unless unique]{singleton}
    \declaretheorem[numbered=unless unique]{couple}
  \end{preamble}
  \begin{body}[gobble=4]
    \begin{couple}
      Marc \& Anne
    \end{couple}
    \begin{singleton}
      Me.
    \end{singleton}
    \begin{couple}
      Buck \& Britta
    \end{couple}
  \end{body}
  \end{source}
  \begin{result}
    \begin{couple}
      Marc \& Anne
    \end{couple}
    \begin{singleton}
      Me.
    \end{singleton}
    \begin{couple}
      Buck \& Britta
    \end{couple}
  \end{result}

  \section{Frilly references}

  In case you didn't know, you should: \pkg{hyperref}, \pkg{nameref} and
  \pkg{cleveref} offer ways of ``automagically'' knowing that
  |\label{foo}| was inside a theorem, so that a reference adds the
  string ``Theorem''. This is all done for you, but there's one catch: you
  have to tell \thmtools\ what the name to add is. By default, it will use
  the title of the theorem, in particular, it will be uppercased.
  (This happens to match the guidelines of all
  publishers I have encountered.) But there is an alternate spelling
  available, denoted by a capital letter, and in any case, if you use
  \pkg{cleveref}, you should give two values separated by a comma, because
  it will generate plural forms if you reference many theorems in one
  |\cite|.
  \begin{source}
    \begin{preamble}[gobble=6]
      \usepackage{amsthm, thmtools}
      \usepackage{
        nameref,%\nameref
        hyperref,%\autoref
        % n.b. \Autoref is defined by thmtools
        cleveref,% \cref
        % n.b. cleveref after! hyperref
      }
      \declaretheorem[name=Theorem, 
        refname={theorem,theorems},
        Refname={Theorem,Theorems}]{callmeal}
    \end{preamble}
    \begin{body}[gobble=6]
      \begin{callmeal}[Simon]\label{simon}
        One
      \end{callmeal}
      \begin{callmeal}\label{garfunkel}
        and another, and together, 
        \autoref{simon}, ``\nameref{simon}'',
        and \cref{garfunkel} are referred 
        to as \cref{simon,garfunkel}.
        \Cref{simon,garfunkel}, if you are at 
        the beginning of a sentence.
      \end{callmeal}
    \end{body}
  \end{source}
  \begin{result}
      \begin{callmeal}[Simon]\label{simon}
        One
      \end{callmeal}
      \begin{callmeal}\label{garfunkel}
        and another, and together, \autoref{simon}, ``\nameref{simon}'',
        and \cref{garfunkel} are referred to as \cref{simon,garfunkel}.
        \Cref{simon,garfunkel}, if you are at the beginning of a sentence.
      \end{callmeal}
  \end{result}

  \section{Styling theorems}
  
  The major backends provide a command |\theoremstyle| to switch
  between looks of theorems. This is handled as follows:
  \begin{source}
  \begin{preamble}[gobble=4]
    \usepackage{amsthm}
    \usepackage{thmtools}
    \declaretheorem[style=remark]{remark}
    \declaretheorem{Theorem}
  \end{preamble}
  \begin{body}[gobble=4]
    \begin{Theorem}
      This is a theorem.
    \end{Theorem}
    \begin{remark}
      Note how it still retains the default style, `plain'.
    \end{remark}
  \end{body}
  \end{source}
  \begin{result}
    \begin{Theorem}
      This is a theorem.
    \end{Theorem}
    \begin{remark}
      Note how it still retains the default style, `plain'.
    \end{remark}
  \end{result}
  
  Thmtools also supports the shadethm and thmbox packages:
  \begin{source}
  \begin{preamble}[gobble=4]
    \usepackage{amsthm}
    \usepackage{thmtools}
    \usepackage[dvipsnames]{xcolor}
    \declaretheorem[shaded={bgcolor=Lavender,
      textwidth=12em}]{BoxI}
    \declaretheorem[shaded={rulecolor=Lavender,
      rulewidth=2pt, bgcolor={rgb}{1,1,1}}]{BoxII}
  \end{preamble}
  \begin{body}[gobble=4]
    \begin{BoxI}[Euclid]
      For every prime $p$, there is a prime $p'>p$.
      In particular, there are infinitely many primes.
    \end{BoxI}
    \begin{BoxII}[Euclid]
      For every prime $p$, there is a prime $p'>p$.
      In particular, there are infinitely many primes.
    \end{BoxII}
  \end{body}
  \end{source}
  \begin{result}
    \begin{BoxI}
      For every prime $p$, there is a prime $p'>p$.
      In particular, there are infinitely many primes.
    \end{BoxI}
    \begin{BoxII}
      For every prime $p$, there is a prime $p'>p$.
      In particular, there are infinitely many primes.
    \end{BoxII}
  \end{result}

  As you can see, the color parameters can take two forms: it's either the
  name of a color that is already defined, without curly braces, or it can
  start with a curly brace, in which case it is assumed that
  |\definecolor{colorname}|$\langle$\textsl{what you said}$\rangle$ will be
  valid \LaTeX\ code. In our case, we use the rbg model to manually specify
  white. (Shadethm's default value is some sort of gray.)
  
  For the thmbox package, use the thmbox key:
  \begin{source}
  \begin{preamble}[gobble=4]
    \usepackage{amsthm}
    \usepackage{thmtools}
    \declaretheorem[thmbox=L]{boxtheorem L}
    \declaretheorem[thmbox=M]{boxtheorem M}
    \declaretheorem[thmbox=S]{boxtheorem S}
  \end{preamble}
  \begin{body}[gobble=4]
    \begin{boxtheorem L}[Euclid]
      For every prime $p$, there is a prime $p'>p$.
      In particular, there are infinitely many primes.
    \end{boxtheorem L}
    \begin{boxtheorem M}[Euclid]
      For every prime $p$, there is a prime $p'>p$.
      In particular, there are infinitely many primes.
    \end{boxtheorem M}
    \begin{boxtheorem S}[Euclid]
      For every prime $p$, there is a prime $p'>p$.
      In particular, there are infinitely many primes.
    \end{boxtheorem S}
  \end{body}
  \end{source}
  \begin{result}
    \begin{boxtheorem L}[Euclid]
      For every prime $p$, there is a prime $p'>p$.
      In particular, there are infinitely many primes.
    \end{boxtheorem L}
    \begin{boxtheorem M}[Euclid]
      For every prime $p$, there is a prime $p'>p$.
      In particular, there are infinitely many primes.
    \end{boxtheorem M}
    \begin{boxtheorem S}[Euclid]
      For every prime $p$, there is a prime $p'>p$.
      In particular, there are infinitely many primes.
    \end{boxtheorem S}
  \end{result}

  Note that for both thmbox and shaded keys, it's quite possible they will not
  cooperate with a style key you give at the same time.

  \subsection{Declaring new theoremstyles}
  
  \Thmtools\ also offers a new command to define new theoremstyles. It is
  partly a frontend to the |\newtheoremstyle| command of \pkg{amsthm} or
  \pkg{ntheorem}, but it offers (more or less successfully) the settings of both to 
  either. So we are talking about the same things, consider the sketch in
  \autoref{fig:params}. To get a result like that, you would use something
  like
  \begin{source}
  \begin{preamble}[gobble=4]
    \declaretheoremstyle[
      spaceabove=6pt, spacebelow=6pt,
      headfont=\normalfont\bfseries,
      notefont=\mdseries, notebraces={(}{)},
      bodyfont=\normalfont,
      postheadspace=1em,
      qed=\qedsymbol
    ]{mystyle}
    \declaretheorem[style=mystyle]{styledtheorem}
  \end{preamble}
  \begin{body}[gobble=4]
    \begin{styledtheorem}[Euclid]
      For every prime $p$\dots
    \end{styledtheorem}
  \end{body}
  \end{source}
  \begin{result}
    \begin{styledtheorem}[Euclid]
      For every prime $p$\dots
    \end{styledtheorem}
  \end{result}
  Again, the defaults are reasonable and you don't
  have to give values for everything.
  
  There is one important thing you cannot see in this example: there are
  more keys you can pass to |\declaretheoremstyle|: if \thmtools\ cannot
  figure out at all what to do with it, it will pass it on to the
  |\declaretheorem| commands that use that style. For example, you may use
  the boxed and shaded keys here.
  
  To change the order in which title, number and note appear, there is a key
  headformat. Currently, the values ``margin'' and ``swapnumber'' are
  supported. The daring may also try to give a macro here that uses the
  commands |\NUMBER|, |\NAME| and |\NOTE|. 
  You cannot circumvent the fact
  that headpunct comes at the end, though, nor the fonts and braces you
  select with the other keys.
  
  \begin{figure}\centering
    % please don't make me touch this picture ever again.
    \fbox{
    \begin{minipage}{0.618\textwidth}\Large
    \tikzset{font=\normalfont\small\sffamily\itshape,y=12pt,>=latex}
    \noindent which resulted in the following insight:
    \par
    \tikz{\draw[|<->|] (0,-1)--(0,1); 
      \draw[anchor=west] (0,0) node {spaceabove};}
    \par
    \tikz{\draw[|<->|] (-1,0)--(1,0); 
      \draw[anchor=south] (0,0) node {headindent};}
    \textbf{Theo%
    \smash{\clap{\tikz{\draw (0,0)--(0,1.2) node [anchor=south]{headfont};}}}%
      rem 1.2 
      (\smash{\rlap{\tikz{\draw (0,0)--(0,2.5) node [anchor=base west]{notebraces};}}}%
       Euc%
\smash{\clap{\tikz{\draw (0,0)--(0,1.2) node [anchor=south]{notefont};}}}%
       lid)%
      .\smash{\rlap{\tikz{\draw (0,0)--(0,1.2) node [anchor=south west]{headpunct};}}}}%
      \tikz{\draw[|<->|] (-1,0)--(1,0); 
      \draw[anchor=south] (0,0) node {postheadspace};}
    For every prime $p$, there is a prime $p'>p$.
    In particular, the list of primes,
    $2,3,5,7,\dots$,
    is infinite.
   \hfill{\tikz{\draw[anchor=north east] (0,0) node{qed}; }}$\Box$
    \par
    \tikz{\draw[|<->|] (0,-1)--(0,1); 
      \draw[anchor=west] (0,0) node {spacebelow};}
    \par
   
   As a consequence, lorem ipsum dolor sit amet frob-%nicate foo
%   paret.
  \end{minipage}
  }
    \caption{Settable parameters of a theorem style.}
    \label{fig:params}
  \end{figure}

  \section{Repeating theorems}
  
  Sometimes, you want to repeat a theorem you have given in full earlier,
  for example you either want to state your strong result in the
  introduction and then again in the full text, or you want to re-state a
  lemma in the appendix where you prove it. For example, I lied about
  \autoref{thm:euclid} on p.\,\pageref{thm:euclid}: the true code used was
  \begin{source}
    \begin{preamble}[gobble=6]
      \usepackage{thmtools, thm-restate}
      \declaretheorem{theorem}
    \end{preamble}
    \begin{body}[gobble=6]
      \begin{restatable}[Euclid]{theorem}{firsteuclid}
        \label{thm:euclid}%
        For every prime $p$, there is a prime $p'>p$.
        In particular, the list of primes,
        \begin{equation}\label{eq:1}
          2,3,45,7,\dots
        \end{equation}
        is infinite.
      \end{restatable}
    \end{body}
    and to the right, I just use
    \begin{body}[gobble=6]
      \firsteuclid*
      \vdots
      \firsteuclid*
    \end{body}
  \end{source}
  \begin{result}
    \firsteuclid*
    \vdots
    \firsteuclid*
  \end{result}
  
  Note that in spite of being a theorem-environment, it gets number one all
  over again. Also, we get equation number~\eqref{eq:1} again. The star in 
  |\firsteuclid*| tells thmtools that it should redirect the label
  mechanism, so that this reference: \autoref{thm:euclid} points to
  p.\,\pageref{thm:euclid}, where the unstarred environment is used. (You can
  also use a starred environment and an unstarred command, in which case the
  behaviour is reversed.) Also, if you use \pkg{hyperref}, the links will lead you
  to the unstarred occurence.
  
  Just to demonstrate that we also handle more involved cases, I repeat
  another theorem here, but this one was numbered within its section: note
  we retain the section number which does not fit the current section:
  \begin{source}
    \begin{body}
      \euclidii*
    \end{body}
  \end{source}
  \begin{result}
    \euclidii*
  \end{result}

  
  \section{Lists of theorems}
  
  To get a list of theorems with default formatting, just use
  |\listoftheorems|:
  \begin{source}
    \begin{body}[gobble=6]
      \listoftheorems
    \end{body}
  \end{source}
  \begin{result}
    \let\chapter\section
    \let\clearpage\relax
    \listoftheorems
  \end{result}
  
  Not everything might be of the same importance, so you can filter out
  things by environment name:
  \begin{source}
    \begin{body}[gobble=6]
      \listoftheorems[ignoreall, 
        show={theorem,Theorem,euclid}]
    \end{body}
  \end{source}
  \begin{result}
    \let\chapter\section
    \let\clearpage\relax
    \listoftheorems[ignoreall, show={theorem,Theorem,euclid}]
  \end{result}
  
  And you can also restrict to those environments that have an optional
  argument given. Note that two theorems disappear compared to the previous
  example. You could also say just ``onlynamed'', in which case it will
  apply to \emph{all} theorem environments you have defined.
  \begin{source}
    \begin{body}[gobble=6]
      \listoftheorems[ignoreall, 
        onlynamed={theorem,Theorem,euclid}]
    \end{body}
  \end{source}
  \begin{result}
    \let\chapter\section
    \let\clearpage\relax
    \listoftheorems[ignoreall, onlynamed={theorem,Theorem,euclid}]
  \end{result}

  As might be expected, the heading given is defined in |\listtheoremname|.

  \section{Extended arguments to theorem environments}
  
  Usually, the optional argument of a theorem serves just to give a note
  that is shown in the theorem's head. \Thmtools\ allows you to have a 
  key-value list here as well. The following keys are known right now:
  \begin{description}
    \item[name] This is what used to be the old argument. It usually holds
    the name of the theorem, or a source. This key also accepts an
    \emph{optional} argument, which will go into the list of theorems. Be
    aware that since we already are within an optional argument, you have to
    use an extra level of curly braces:
    |\begin{theorem}[{name=[Short name]A long name,...}]|
    \item[label] This will issue a |\label| command after the head. Not very
    useful, more of a demo.
    \item[continues] Saying |continues=foo| will cause the number that is
    given to be changed to |\ref{foo}|, and a text is added to the note.
    (The exact text is given by the macro |\thmcontinues|, which takes the
    label as its argument.)
    \item[restate] Saying |restate=foo| will hopefully work like
    wrapping this theorem in a restatable environment. (It probably still fails
    in cases that I didn't think of.) This key also accepts an optional
    argument: when restating, the restate key is replaced by this argument,
    for example, |restate=[name=Boring rehash]foo| will result in a
    different name. (Be aware that it is possible to give the same key
    several times, but I don't promise the results. In case of the name key,
    the names happen to override one another.)
  \end{description}
  \begin{source}
    \begin{body}[gobble=6]
      \begin{theorem}[name=Keyed theorem, 
        label=thm:key]
        This is a 
        key-val theorem.
      \end{theorem}
      \begin{theorem}[continues=thm:key]
        And it's spread out.
      \end{theorem}
    \end{body}
  \end{source}
  \begin{result}
    \begin{theorem}[name=Keyed theorem, 
        label=thm:key]
        This is a key-val theorem.
    \end{theorem}
    \begin{theorem}[continues=thm:key]
        And it's spread out.
    \end{theorem}
  \end{result}

  

  \chapter{\Thmtools for the extravagant}\label{cha:extravagant}

  This chapter will go into detail on the slightly more technical offerings
  of this bundle. In particular, it will demonstrate how to use the general
  hooks provided to extend theorems in the way you want them to behave.
  Again, this is done mostly by some examples.
  
  \section{Understanding \thmtools' extension mechanism}

  \Thmtools\ draws most of its power really only from one feature:
  the |\newtheorem| of the backend will, for example, create
  a theorem environment, i.e. the commands |\theorem| and
  |\endtheorem|. To add functionality, four places immediately
  suggest themselves: ``immediately before'' and ``immediately after'' those
  two.
  
  There are two equivalent ways of adding code there: one is to call
  |\addtotheorempreheadhook| and its brothers and sisters
  |...postheadhook|, |...prefoothook|
  and |...postfoothook|.
  All of these take an \emph{optional} argument, the name of the
  environment, and the new code as a mandatory argument. The environment is 
  optional because there  is also a set of ``generic'' hooks added to every
  theorem that you define.
  
  The other way is to use the keys |preheadhook| et al. in your
  |\declaretheorem|. (There is no way of accessing the generic
  hook in this way.)
  
  The hooks are arranged in the following way: first the specific prehead,
  then the generic one. Then, the original |\theorem| (or
  whatever) will be called. Afterwards, first the specific posthead again,
  then the generic one. (This means that you cannot wrap the head alone in
  an environment this way.) At the end of the theorem, it is the other way
  around: first the generic, then the specific, both before and after that
  |\endtheorem|. This means you can wrap the entire theorem
  easily by adding to the prehead and the postfoot hooks. Note that
  \thmtools\ does not look inside |\theorem|, so you cannot get
  inside the head formatting, spacing, punctuation in this way.
  
  In many situations, adding static code will not be enough. Your code can
  look at |\thmt@envname|, |\thmt@thmname| and 
  |\thmt@optarg|, which will contain the name of the environment,
  its title, and, if present, the optional argument (otherwise, it is
  |\@empty|).
  \emph{However}, you should not make assumptions about the optional
  argument in the preheadhook: it might still be key-value, or it might
  already be what will be placed as a note. (This is because the key-val
  handling itself is added as part of the headkeys.)


  \section{Case in point: the shaded key}
  
  Let us look at a reasonably simple example: the shaded key, which we've
  already seen in the first section. You'll observe that we run into a
  problem similar to the four-hook mess: your code may either want to modify
  parameters that need to be set beforehand, or it wants to modify the 
  environment after it has been created. To hide this from the user, the
  code you define for the key is actually executed twice, and
  |\thmt@trytwice{A}{B}| will execute A on the first pass, and B
  on the second. Here, we want to add to the hooks, and the hooks are only
  there in the second pass.
  
  \DocInput{thmdef-shaded.dtx}

  \section{Case in point: the thmbox key}

  \DocInput{thmdef-thmbox.dtx}

  \section{Case in point: the mdframed key}
  \DocInput{thmdef-mdframed.dtx}
  
  \section{How \thmtools\ finds your extensions}

  Up to now, we have discussed how to write the code that adds functionality
  to your theorems, but you don't know how to activate it yet. 
  Of course, you can put it in your preamble, likely embraced by
  |\makeatletter| and |\makeatother|, because you are
  using internal macros with @ in their name (viz.,
  |\thmt@envname| and friends). You can also put them into a
  package (then, without the |\makeat...|), 
  which is simply a file ending in .sty put somewhere that \LaTeX\ can find
  it, which can then be laoded with |\usepackage|.
  To find out where exactly that is, and if you'd need to update
  administrative helper files such as a filename database FNDB, 
  please consult the documentation of your \TeX\ distribution.
  
  Since you most likely want to add keys as well, there is a shortcut that
  \thmtools\ offers you: whenever you use a key |key| in a
  |\declaretheorem| command, and \thmtools\ doesn't already know
  what to do with it, it will try to |\usepackage{thmdef-key}| and
  evaluate the key again. (If that doesn't work, \thmtools\ will cry
  bitterly.)
  
  For example, there is no provision in \thmtools\ itself that make the
  |shaded| and |thmbox| keys described above special:
  in fact, if you want to use a different package to create frames, you just
  put a different |thmdef-shaded.sty| into a preferred texmf tree.
  Of course, if your new package doesn't offer the old keys, your old
  documents might break!
  
  The behaviour for the keys in the style definition is slightly different:
  if a key is not known there, it will be used as a ``default key'' to every
  theorem that is defined using this style. For example, you can give the
  |shaded| key in a style definition.
  
  Lastly, the key-val arguments to the theorem environments themselves need
  to be loaded manually, not least because inside the document it's too late
  to call |\usepackage|.

  \chapter{\Thmtools for the completionist}\label{cha:reference}

  This will eventually contain a reference to all known keys, commands, etc.


  \section{Known keys to \texttt{\textbackslash declaretheoremstyle}}
  \def\keydefaultcontext{declaretheoremstyle}

  N.b. implementation for \pkg{amsthm} and \pkg{ntheorem} is separate for
  these, so if it doesn't work for \pkg{ntheorem}, try if it works with
  \pkg{amsthm}, which in general supports more things.

  Also, all keys listed as known to |\declaretheorem| are valid.

  \key{spaceabove} Value: a length. Vertical space above the theorem,
  possibly discarded if the theorem is at the top of the page.

  \key{spacebelow} Value: a length. Vertical space after the theorem,
  possibly discarded if the theorem is at the top of the page.

  \key{headfont} Value: \TeX\ code. Executed just before the head of the
  theorem is typeset, inside a group. Intended use it to put font switches here.

  \key{notefont}
    Value: \TeX\ code. Executed just before the note in the head is typeset, inside a group. 
    Intended use it to put font switches here. Formatting also applies to
    the braces around the note.
    Not supported by \pkg{ntheorem}.

  \key{bodyfont} 
    Value: \TeX\ code. Executed before the begin part of the theorem ends,
    but before all afterheadhooks. Intended use it to put font switches here.
  
  \key{headpunct}
    Value: \TeX\ code, usually a single character. Put at the end of the
    theorem's head, prior to linebreaks or indents.
  
  \key{notebraces}
    Value: Two characters, the opening and closing symbol to use around a
    theorem's note.
    (Not supported by \pkg{ntheorem}.)

  \key{postheadspace}
    Value: a length. Horizontal space inserted after the entire head of the
    theorem, before the body. Does probably not apply (or make sense) for
    styles that have a linebreak after the head.

  \key{headformat}
    Value: \LaTeX\ code using the special placeholders |\NUMBER|, |\NAME|
    and |\NOTE|, which correspond to the (formatted, including the braces
    for |\NOTE| etc.) three parts of a theorem's head. This can be used to
    override the usual style ``1.1 Theorem (Foo)'', for example to let the
    numbers protude in the margin or put them after the name.
    
    Additionally, a number of keywords are allowed here instead of \LaTeX\
    code:
    \begin{description}
      \item[margin] Lets the number protude in the (left) margin.
      \item[swapnumber] Puts the number before the name. Currently
      not working so well for unnumbered theorems.
      \item[] \emph{This list is likely to grow}
    \end{description}

  \key{headindent}
    Value: a length. Horizontal space inserted before the head. Some
    publishers like |\parindent| here for remarks, for example.


  \section{Known keys to \texttt{\textbackslash declaretheorem}}
  \def\keydefaultcontext{declaretheorem}

  \key{parent}
    Value: a counter name. The theorem will be reset whenever that counter
    is incremented. Usually, this will be a sectioning level, |chapter| or
    |section|.

  \key{numberwithin}
    Value: a counter name. The theorem will be reset whenever that counter
    is incremented. Usually, this will be a sectioning level, |chapter| or
    |section|.
    (Same as parent.)

  \key{within}
    Value: a counter name. The theorem will be reset whenever that counter
    is incremented. Usually, this will be a sectioning level, |chapter| or
    |section|.
    (Same as parent.)

  
  \key{sibling}
    Value: a counter name. The theorem will use this counter for numbering.
    Usually, this is the name of another theorem environment.
    
  \key{numberlike}
    Value: a counter name. The theorem will use this counter for numbering.
    Usually, this is the name of another theorem environment.
    (Same as sibling.)
  
  \key{sharenumber}
    Value: a counter name. The theorem will use this counter for numbering.
    Usually, this is the name of another theorem environment.
    (Same as sibling.)

  
  \key{title}
    Value: \TeX\ code. The title of the theorem. Default is the name of the
    environment, with |\MakeUppercase| prepended. You'll have to give
    this if your title starts with a accented character, for example.

  \key{name}
    Value: \TeX\ code. The title of the theorem. Default is the name of the
    environment, with |\MakeUppercase| prepended. You'll have to give
    this if your title starts with a accented character, for example.
    (Same as title.)

  \key{heading}
    Value: \TeX\ code. The title of the theorem. Default is the name of the
    environment, with |\MakeUppercase| prepended. You'll have to give
    this if your title starts with a accented character, for example.
    (Same as title.)
  
  \key{numbered}
    Value: one of the keywords |yes|, |no| or |unless unique|. The theorem
    will be numbered, not numbered, or only numbered if it occurs more than
    once in the document. (The latter requires another \LaTeX\ run and will
    not work well combined with |sibling|.)
  
  \key{style}
    Value: the name of a style defined with |\declaretheoremstyle| or
    |\newtheoremstyle|. The theorem will use the settings of this style.
  
  \key{preheadhook}
    Value: \LaTeX\ code. This code will be executed at the beginning of the
    environment, even before vertical spacing is added and the head is
    typeset. However, it is already within the group defined by the
    environment.
  
  \key{postheadhook}
    Value: \LaTeX\ code. This code will be executed after the call to the
    original begin-theorem code. Note that all backends seem to delay
    typesetting the actual head, so code here should probably enter
    horizontal mode to be sure it is after the head, but this will change
    the spacing/wrapping behaviour if your body starts with another list.

  \key{prefoothook}
    Value: \LaTeX\ code. This code will be executed at the end of the body
    of the environment.
    
  \key{postfoothook}
    Value: \LaTeX\ code. This code will be executed at the end of the
    environment, even after eventual vertical spacing, but still within the
    group defined by the environment.
  
  \key{refname}
    Value: one string, or two string separated by a comma (no spaces). This
    is the name of the theorem as used by |\autoref|, |\cref| and friends. If it is
    two strings, the second is the plural form used by |\cref|. Default
    value is the value of |name|, i.e. usually the environment name, with
    \MakeUppercase.

  \key{Refname}
    Value: one string, or two string separated by a comma (no spaces). This
    is the name of the theorem as used by |\Autoref|, |\Cref| and friends. If it is
    two strings, the second is the plural form used by |\Cref|. This can be
    used for alternate spellings, for example if your style requests no
    abbreviations at the beginning of a sentence. No default.

  
  \key{shaded}
    Value: a key-value list, where the following keys are possible:
    \begin{description}
      \item[textwidth]
        The linewidth within the theorem.
      \item[bgcolor]
        The color of the background of the theorem. Either a color name or a
        color spec as accepted by |\definecolor|, such as |{gray}{0.5}|.
      \item[rulecolor]
        The color of the box surrounding the theorem. Either a color name or
        a color spec.
      \item[rulewidth]
        The width of the box surrounding the theorem.
      \item[margin]
        The length by which the shade box surrounds the text.
    \end{description}
  
  \key{thmbox}
    Value: one of the characters L, M and S; see examples above.
  

  \section{Known keys to in-document theorems}
  \def\keydefaultcontext{theorem}

  \key{label} Value: a legal |\label| name.
  Issues a |\label| command after the theorem's head.

  \key{name} Value: \TeX\ code that will be typeset.
  What you would have put in the optional argument in the
  non-keyval style, i.e. the note to the head. This is \emph{not} the same
  as the name key to |\declaretheorem|, you cannot override that from within
  the document.
  
  \key{listhack} Value: doesn't matter. (But put something to trigger
  key-val behaviour, maybe listhack=true.) Linebreak styles in \pkg{amsthm}
  don't linebreak if they start with another list, like an |enumerate|
  environment. Giving the |listhack| key fixes that. \emph{Don't} give this
  key for non-break styles, you'll get too little vertical space! (Just use
  |\leavevmode| manually there.)
  An all-around listhack that handles both situations might come in a
  cleaner rewrite of the style system.

  \section{Restatable -- hints and caveats}

  TBD.
  \begin{itemize}
    \item Some counters are saved so that the same values appear when you
    re-use them. The list of these counters is stored in the macro
    |\thmt@innercounters| as a comma-separated list without spaces; default: equation.

    \item To preserve the influence of other counters (think: equation
    numbered per section and recall the theorem in another section), we need
    to know all macros that are used to turn a counter into printed output.
    Again, comma-separated list without spaces, without leading backslash, stored as
    |\thmt@counterformatters|. Default:
    |@alph,@Alph,@arabic,@roman,@Roman,@fnsymbol|
    All these only take
    the \LaTeX\ counter |\c@foo| as arguments. If you bypass this and use
    |\romannumeral|, your numbers go wrong and you get what you deserve.
    Important if you have very strange numbering, maybe using greek letters
    or somesuch.

    \item I think you cannot have one stored counter within another one's
    typeset representation. I don't think that ever occurs in reasonable
    circumstances, either. Only one I could think of: multiple subequation
    blocks that partially overlap the theorem. Dude, that doesn't even nest.
    You get what you deserve.
    
    \item |\label| and \pkg{amsmath}'s |\ltx@label| are disabled inside the
    starred execution. Possibly, |\phantomsection| should be disabled as
    well?
  \end{itemize}

  \appendix
  

  \chapter{\Thmtools for the morbidly curious}\label{cha:sourcecode}

  This chapter consists of the implementation of Thmtools, in case you
  wonder how this or that feature was implemented. Read on if you want a
  look under the bonnet, but you enter at your own risk, and bring an oily
  rag with you.

  \section{Core functionality}
  
  \subsection{The main package}
  \DocInput{thmtools.dtx}

  \subsection{Adding hooks to the relevant commands}
  \DocInput{thm-patch.dtx}

  \subsection{The key-value interfaces}
  \DocInput{thm-kv.dtx}

  \subsection{Lists of theorems}
  \DocInput{thm-listof.dtx}

  \subsection{Re-using environments}
  \DocInput{thm-restate.dtx}

  \subsection{Fixing autoref and friends}  
  \DocInput{thm-autoref.dtx}

  \section{Glue code for different backends}
  
  \subsection{amsthm}
  \DocInput{thm-amsthm.dtx}

  \subsection{beamer}
  \DocInput{thm-beamer.dtx}

  \subsection{ntheorem}
  \DocInput{thm-ntheorem.dtx}
  
  \section{Generic tools}
  
  \subsection{A generalized argument parser}
  \DocInput{parseargs.dtx}
  
  \subsection{Different counters sharing the same register}
  \DocInput{aliasctr.dtx}
  
  \subsection{Tracking occurences: none, one or many}
  \DocInput{unique.dtx}

  

      
\end{document}
