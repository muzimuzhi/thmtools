\documentclass{article}

\usepackage{amsmath, amsthm}


\usepackage{
  thm-listof,
  thm-restate,
  thm-autoref,
  thm-kv,
  }

\usepackage{hyperref}

\declaretheorem[unnumbered,
   title={Zorn's Lemma}]{zl}
%\newtheorem*{zl}{Zorn's Lemma}
\declaretheorem[numberwithin=section]{theorem}
%\newtheorem{theorem}{Theorem}[section]
\declaretheorem[sibling=theorem]{lemma}
%\newtheorem{lemma}[theorem]{Lemma}
\declaretheorem[numberlike=lemma]{axiom}
%\newtheorem{axiom}[lemma]{Axiom}


\begin{document}
  \section{Introduction}
  
  In this dummy document, we will show important things. One very important
  insight is
  \begin{lemma}[Zorn]
    If every chain in $X$ is bounded, $X$ has a maximal element.
    
    (Here, $X$ is a set system.)
  \end{lemma}
  
  This lemma is so important that it's a fixed name:
  \begin{restatable}{zl}{zornslemma}
    If every chain in $X$ is bounded, $X$ has a maximal element.
    
    (Here, $X$ is a set system.)
  \end{restatable}
  
  We will conclude in important theorem from this:
  \begin{restatable}[Well-ordering]{theorem}{wohlordnung}\label{thm:order}
    Every set is well-ordered.
  \end{restatable}
  
  %\show\wohlordnung

  \section{Main}
  
  Here, we will prove \wohlordnung which first appeared
  as~\autoref{thm:order} on page~\pageref{thm:order} and is 
  actually equivalent to
  \zornslemma
  
  Another equivalent formulation is
  \begin{axiom}[Axiom of Choice]
    If you have a non-empty set, you can take an element out of it.
  \end{axiom}
  
  \section{Conclusion}
  
  To remind you, these are the theorems that occur in this document,
  ignoring Lemmas:
  
  \ignoretheorems{lemma}
  \listoftheorems
  
\end{document}
